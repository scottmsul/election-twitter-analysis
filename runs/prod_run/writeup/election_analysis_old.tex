\documentclass[12pt]{extarticle}

\title{
{\LARGE \bfseries Predicting the Upcoming Primaries using Twitter Data}
}

\author{Scott Sullivan}

\begin{document}

\maketitle

\begin{abstract}
This is the abstract.

\end{abstract}

\newpage

\section{Introduction}
Twitter offers an interesting dataset for analyzing and predicting election results.

\section{Methods}
\subsection{Downloading and Organizing Tweets}
First, tweets were downloaded and categorized as either pro-Sanders or pro-Clinton.
Rather than using sentiment analysis, each tweet was filtered using hashtags.
Although this is a particularly strong filter (resulting in fewer tweets than simply matching candidate names), the sentiment of each tweet is more unambiguous.
The hashtags used in the analysis are shown in the table below.
\\
\\
\begin{centering}
	\begin{tabular}{|c|c|} \hline
    Candidate & Positive hashtags \\ \hline 
    Bernie Sanders & \#feelthebern, \#bernieorbust, \#stillsanders, \#bernie2016 \\ \hline
    Hillary Clinton & \#hillary2016, \#imwithher, \#clintonfoundation \\ \hline
  \end{tabular}
\end{centering}
\\
\\
For the analysis, estimates for candidate support were based on number of users rather than number of tweets.
This removes any bias towards users who tweet more often.
Each user's preferred candidate was calculated by tallying the number of pro-Clinton tweets versus the number of pro-Sanders tweets.
\\
\indent
Next, the users were separated into different states.
It was found that matching a simple regular expression on each user's location could reliably retrieve a user's state for about a third of all tweets.
The algorithm matched either the state's full name, or the state's abbreviation at the end of the user's location.
For Washington, the full-name match was excluded in order to avoid ambiguity with the District of Columbia.

\subsection{Predicting Future Delegates}
In order to predict future delegates, results from past primaries can be used in a fitting function.
For the 44 states whose primary has already occured, the number of votes for each candidate was downloaded from Google.
There exist several possible biases which must be accounted before selecting a fitting approach.
\\
\indent
The first is the bias between each candidate's Twitter supporters and voters.
For example, if Sanders supporters are younger, and Twitter users are also younger, than Sanders would be overestimated in the Twitter data relative to the election results.
To take this bias into account, the ratio between voters and Twitter supporters was fitted using past primaries and used to predict future primaries.
If $V_S$ is the number of Sanders voters in a state, and $T_S$ is the number of Sanders Twitter supporters in a state, the ratio $R_S = V_S / T_S$ can be defined as the ratio between voters and Twitter supporters for Sanders in some state.
Similarly, $R_C$ can be defined for Clinton instead of Sanders.
\\
\indent
The second bias to take into account is the varying degree of total voters in each state.
Some states have caucuses, others have closed primaries, and others have open primaries.
The different rules in each state affect the total amount of voter participation.
Therefore, simply averaging $R_S$ and $R_C$ won't work.
However, a new ratio can be defined as $R_R = R_S / R_C$.
This ratio of ratios should be similar across all states, since it is independent of total voter participation.
The ratio of ratios can be fitted by averaging data from past primaries, then used to calculate future primary results.
\\
\indent
If the ratio of ratios is known, and the number of Twitter supporters for each candidate is known, then the number of received delegates can be estimated as follows.
The ratio of votes is given by
\begin{equation}
\frac{V_S}{V_C} = \frac{R_S T_S}{R_C T_C} = R_R \frac{T_S}{T_C}.
\end{equation}
This should be proportional to the ratio of delegates,
\begin{equation}
\frac{D_S}{D_C} = R_R \frac{T_S}{T_C}.
\end{equation}
Let D be the number of delegates in a state whose primary hasn't happened yet.
By normalizing $D_S + D_C = D$, we can predict the number of delegates obtained by each candidate.
Let $x = D_S / D_C$.
The estimated delegates are given by
\begin{equation}
D_S = \frac{x}{1+x} D,
\end{equation}
\begin{equation}
D_C = \frac{1}{1+x} D.
\end{equation}

\subsection{Estimating Uncertainties}
To estimate uncertainties, the Twitter feed from each state was assumed to come from a Poisson distribution.
Therefore, the uncertainties in $T_S$ and $T_C$ are $\sigma_{T_S} = \sqrt{T_S}$ and $\sigma_{T_C} = \sqrt{T_C}$, respectively.
For each state, the uncertainties in $R_S$ and $R_C$ were propagated using partial derivatives,
\begin{equation}
\sigma_R = \frac{R}{V} \sigma_V.
\end{equation}
The uncertainties for each state's ratio of ratios were propagated,
\begin{equation}
\sigma_{R_R}^2 = \Big(\frac{R_R}{R_S} \sigma_{R_S}\Big)^2 + \Big(\frac{R_R}{R_C} \sigma_{R_C}\Big)^2
\end{equation}
The average value for $R_R$ was calculated using
\begin{equation}
	R_R = \frac{\Sigma w_i R_{R_i}}{\Sigma w_i},
\end{equation}
where the sums are over each state $i$, and the weights are given by $w_i = 1 / \sigma_{R_i}^2$.
The uncertainty in $R_R$ was calculated using
\begin{equation}
\sigma_{R_R} = \frac{1}{\sqrt{\Sigma w_i}}.
\end{equation}
Finally, the uncertainty in future states was a combination of Poisson error from each state's Twitter users and the uncertainty in $R_R$.
Using $x = T_S / T_C$,
\begin{equation}
	\sigma_x^2 = \Big(\frac{x}{T_C} \sigma_{T_C}\Big)^2 + \Big(\frac{x}{T_S} \sigma_{T_S}\Big)^2
\end{equation}
For each candidate, the delegate uncertainty propagates similarly,
\begin{equation}
\sigma_D = \frac{1}{(1+x)^2} D \sigma_x.
\end{equation}

\section{Results}
Tweets were accumulated over a period of about ten hours, resulting in 40,500 relevant tweets and 5,000 unique users.
Typically there were 50 users per candidate per state, so the Poisson errors were pretty significant.
\\
\indent
The average ratio of ratios was given by
\begin{equation}
R_R = 0.408 \pm 0.016.
\end{equation}
This number essentially tells us Sanders' overrepresentation on Twitter.
For every matching vote between Sanders and Clinton, there should be a ratio of five-to-two in favor of Sanders over Clinton on Twitter.
\\
\indent
The main results are shown in the tables below.
South Dakota was excluded for not having any Clinton Twitter supporters (and only 4 Sanders supporters).
First are the Twitter supporters in the five remaining states.
\\
\\
\begin{centering}
  \begin{tabular}{|l|l|l|}
	  \hline
     State & Sanders Twitter supporters & Clinton Twitter supporters \\
	  \hline
	  \hline
     North Dakota & 4 & 2 \\
     California & 855 & 250 \\
     New Jersey & 67 & 44 \\
     New Mexico & 23 & 10 \\
     Montana & 13 & 2 \\
	  \hline
  \end{tabular}
\end{centering}
\\
\\

Next are the expected delegates in each state.
\\
\\
\begin{centering}
	\begin{tabular}{|l|l|l|l|l|}
	  \hline
  State & Sanders Delegates & Uncertainty & Clinton Delegates & Uncertainty \\
	  \hline
	  \hline
  North Dakota & 8 & 2 & 10 & 1 \\
  California & 277 & 29 & 198 & 16 \\
     New Jersey & 48 & 8 & 78 & 7 \\
     New Mexico & 16 & 5 & 17 & 3 \\
     Montana & 15 & 4 & 6 & 1 \\
	  \hline
  \end{tabular}
\end{centering}
\\
\\
Finally are the total delegates from the five states.
\\
\\
\begin{centering}
	\begin{tabular}{|l|l|l|l|l|}
	  \hline
	  Candidate & Total Delegates & Uncertainty \\
	  \hline
	  \hline
	  Bernie Sanders & 365 & 31 \\
	  \hline
	  Hillary Clinton & 309 & 18 \\
	  \hline
  \end{tabular}
\end{centering}
\\
\\

\begin{conclusion}
Currently Hillary Clinton is ahead by about 270 delegates. 
We can see that Sanders has a pretty solid lead over Clinton in Calfornia, but is losing in New Jersey.
On June 7 Sanders will likely close the gap by about 50 delegates, which is probably not significant enough to overcome Hillary Clinton.
The uncertainties are somewhat large, so some upsets are to be expected.
In addition, there are likely unaccounted systematics between Twitter behavior in different states or regions.
\\
\indent
It would be interesting to attempt a similar analysis in the general election with Donald Trump.
This particular analysis exploited the spaced timings of each primary election, which won't be available in the general election.
However, perhaps votes in past Republican primaries could be used to fit similar ratios.

\end{document}
